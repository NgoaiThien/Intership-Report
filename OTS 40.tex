\section{Máy sấy tiêu bản}
\subsection{Giới thiệu chung}
OTS 40  là dòng bàn sấy tiêu bản chất lượng cao, được thiết kế chuyên dụng cho các phòng thí nghiệm mô học, giải phẫu bệnh và tế bào học.

Trong quy trình kỹ thuật, sau khi mẫu mô được cắt và dàn phẳng trên bể nước, chúng cần được sấy khô để đảm bảo mẫu mô bám chặt vào lam kính và loại bỏ hoàn toàn nước thừa trước khi tiến hành nhuộm màu. OTS 40 đảm nhiệm vai trò này.

Thiết bị nổi bật với thiết kế thon gọn , khả năng gia nhiệt nhanh, phân bố nhiệt đồng đều và diện tích bề mặt tối ưu cho phép xử lý số lượng mẫu lớn cùng lúc.\cite{Catalog_OTS40}
\subsection{Cấu tạo}
\begin{figure}[H]
    \centering
    \includegraphics[width=0.8\linewidth]{ots40_more.png}
    \caption{OTS 40 \cite{Catalog_OTS40}}
    \label{fig:ots40}
\end{figure}
\subsection{Thông số kỹ thuật}
\begin{table}[H]
	\centering
	\renewcommand{\arraystretch}{1.3} % Tăng khoảng cách dòng
	\begin{tabular}{|p{6cm}|p{10cm}|}
		\hline
		% --- OTS 40.1520 ---
		\multicolumn{2}{|l|}{\bfseries Thông số kĩ thuật OTS 40.1520 \cite{Catalog_OTS40}} \\ \hline
		Nhiệt độ: & +30 $^\circ$C đến +99 $^\circ$C \\ \hline
		Kích thước tổng thể (W/D/H): & 150 x 200 x 80 mm \\ \hline
		Cân nặng: & 1.7 kg \\ \hline
		Nguồn điện: & 230 V / 50 – 60 Hz / 120 VA $\rightarrow$ Cat. No. 01-4001-00 \newline 115 V / 50 – 60 Hz / 120 VA $\rightarrow$ Cat. No. 01-4101-00 \\ \hline
		
		% --- OTS 40.1540 ---
		\multicolumn{2}{|l|}{\bfseries Thông số kĩ thuật OTS 40.1540} \\ \hline
		Nhiệt độ: & +30 $^\circ$C đến +99 $^\circ$C \\ \hline
		Kích thước tổng thể (W/D/H): & 150 x 400 x 80 mm \\ \hline
		Cân nặng: & 2.9 kg \\ \hline
		Nguồn điện: & 230 V / 50 – 60 Hz / 300 VA $\rightarrow$ Cat. No. 01-4002-00 \newline 115 V / 50 – 60 Hz / 300 VA $\rightarrow$ Cat. No. 01-4102-00 \\ \hline
		
		% --- OTS 40.2025 ---
		\multicolumn{2}{|l|}{\bfseries Thông số kĩ thuật OTS 40.2025} \\ \hline
		Nhiệt độ: & +30 $^\circ$C đến +99 $^\circ$C \\ \hline
		Kích thước tổng thể (W/D/H): & 200 x 250 x 80 mm \\ \hline
		Cân nặng: & 2.4 kg \\ \hline
		Nguồn điện: & 230 V / 50 – 60 Hz / 250 VA $\rightarrow$ Cat. No. 01-4003-00 \newline 115 V / 50 – 60 Hz / 250 VA $\rightarrow$ Cat. No. 01-4103-00 \\ \hline
		
		% --- OTS 40.2530 ---
		\multicolumn{2}{|l|}{\bfseries Thông số kĩ thuật OTS 40.2530} \\ \hline
		Nhiệt độ: & +30 $^\circ$C đến +99 $^\circ$C \\ \hline
		Kích thước tổng thể (W/D/H): & 250 x 300 x 80 mm \\ \hline
		Cân nặng: & 3.6 kg \\ \hline
		Nguồn điện: & 230 V / 50 – 60 Hz / 350 VA $\rightarrow$ Cat. No. 01-4004-00 \newline 115 V / 50 – 60 Hz / 350 VA $\rightarrow$ Cat. No. 01-4104-00 \\ \hline
		
		% --- OTS 40.3040 ---
		\multicolumn{2}{|l|}{\bfseries Thông số kĩ thuật OTS 40.3040} \\ \hline
		Nhiệt độ: & +30 $^\circ$C đến +99 $^\circ$C \\ \hline
		Kích thước tổng thể (W/D/H): & 300 x 400 x 80 mm \\ \hline
		Cân nặng: & 5.4 kg \\ \hline
		Nguồn điện: & 230 V / 50 – 60 Hz / 550 VA $\rightarrow$ Cat. No. 01-4005-00 \newline 115 V / 50 – 60 Hz / 550 VA $\rightarrow$ Cat. No. 01-4105-00 \\ \hline
	\end{tabular}
\end{table}

\subsection{Lắp đặt}
Trước khi bắt đầu làm việc trên thiết bị, hãy để thiết bị ở vị trí được chỉ định ít nhất 2 giờ trở lên, để điều chỉnh theo nhiệt độ phòng.\\
Thiết bị phải được cài đặt ổn định, ngang bằng và được định hướng theo chiều ngang cơ sở để đảm bảo hoạt động an toàn và đáng tin cậy.\\
Trước khi khởi động thiết bị, vui lòng đảm bảo rằng điện áp nguồn của bạn tương ứng với giá trị được chỉ ra trên thiết bị (ví dụ: đối với thiết bị yêu cầu 240 Volts, phải có 200 - 240 Volts).\cite{manual_OTS40}\\
\subsection{Vận hành}
Có thể bật thiết bị bằng cách nhấn công tắc nguồn vị trí 'I' ở sau thiết bị. Công tắc có đèn báo sẽ sáng lên, cho biết thiết bị đã sẵn sàng để sử dụng.
Nhiệt độ của tấm làm việc có thể điều chỉnh vô cấp theo từng bước 1°C, từ +30°C đến 90°C.\cite{manual_OTS40}\\
Nhiệt độ được điều chỉnh và hiển thị bởi bộ điều khiển nhiệt độ điện tử trên bảng điều khiển phía trước.
\begin{figure}[H]
	\centering
	\includegraphics[width=0.5\linewidth]{temperature.png}
	\label{fig:tem}
\end{figure}
\begin{figure}[H]
	\centering
	\includegraphics[width=0.7\linewidth]{setting.png}
	\label{fig:set}
\end{figure}
Ở chế độ hoạt động bình thường của thiết bị, nhiệt độ thực tế của bề mặt làm việc sẽ được hiển thị liên tục. Màn hình kỹ thuật số hiển thị nhiệt độ cài đặt trong quá trình cài đặt.
\subsubsection*{Cài đặt nhiệt độ}
\begin{enumerate}
	\item Nhấn SET để chuyển sang chế độ " thay đổi nhiệt độ cài đặt".
	\item Nhấn nút Lên và Xuống để thay đổi nhiệt độ.
	\item Nhấn SET để lưu thay đổi.
	
\end{enumerate}
Chế độ thay đổi nhiệt độ sẽ tự động đóng sau 10 giây.
\subsection{Khắc phục sự cố}
\subsubsection{Thay cầu chì}
Để thay cầu chì, hãy tắt thiết bị. Nhấn nút bật/tắt và rút dây nguồn. Sau đó kéo ngăn (1) với các cầu chì ra một cách cẩn thận.
 \begin{figure}[H]
 	\centering
 	\includegraphics[width=0.7\linewidth]{fix.png}
 	\label{fig:fix}
 \end{figure}
 Thay cầu chì bị lỗi bằng cầu chì mới và trượt khay trở lại khe cắm. \cite{manual_OTS40}
\subsection{Bảo dưỡng}
\subsubsection{Hướng dẫn làm sạch}
Bề mặt OTS 40 có thể được làm sạch bằng các chất làm sạch đã được tiêu chuẩn hoá, không làm trầy xước, nếu cần. Không sử dụng các dụng cụ hoặc lưỡi dao có lưỡi sắc nhọn.\\
- Trước khi bắt đầu vệ sinh, luôn tắt thiết bị và kéo dây nguồn.\\
- Quy trình vệ sinh hằng ngày - sau khi kết thúc công việc hằng ngày - kế hoạch vệ sinh bảo trì.\\

