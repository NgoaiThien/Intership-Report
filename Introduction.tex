\chapter{Lời mở đầu}

\section{Lý do và tính cấp thiết của đợt thực tập}
Việc hoàn thành chương trình đào tạo Cử nhân Vật lí không chỉ đòi hỏi nắm vững kiến thức lý thuyết mà còn cần khả năng vận dụng vào thực tiễn. Nhằm gắn kết giữa học tập và thực hành, em đã lựa chọn thực tập tại Công ty TNHH Thương mại Kỹ thuật Phú Nguyên – đơn vị uy tín trong lĩnh vực cung cấp thiết bị và vật tư y tế. Đây là cơ hội để em quan sát, học hỏi và trải nghiệm môi trường làm việc chuyên nghiệp.
Trong quá trình thực tập, em có dịp vận dụng các kiến thức vật lí đã học (như quang học, điện từ học, vật lí hạt nhân) để tìm hiểu nguyên lý hoạt động và ứng dụng của các thiết bị y tế chuyên dụng. Việc tiếp xúc trực tiếp với các thiết bị hiện đại không chỉ giúp em củng cố kiến thức chuyên môn mà còn mở rộng tầm nhìn về vai trò của Vật lí trong y học.\\[0.3 cm]
Bên cạnh đó, thực tập cũng giúp em rèn luyện các kỹ năng mềm quan trọng như giao tiếp, làm việc nhóm và thích ứng với môi trường công việc thực tế. Đây chính là hành trang quý giá để em tự tin hơn, đáp ứng tốt yêu cầu ngày càng cao của thị trường lao động và tạo nền tảng vững chắc cho sự nghiệp sau này.\\[0.3 cm]
\section{Mục tiêu của đợt thực tập}
Mục tiêu của đợt thực tập được xác định rõ ràng trên ba phương diện chính: kiến thức, kỹ năng và thái độ. \\[0.3 cm]
Mục tiêu về kiến thức, hiểu rõ cấu tạo, nguyên lý vật lí và cách thức vận hành của một số vật tư tiêu hao, thiết bị y tế cơ bản như máy cắt tiêu bản và một số thiết bị phân tích y tế khác. Nắm bắt các quy định và tiêu chuẩn về an toàn bức xạ, an toàn thiết bị để đảm bảo quá trình sử dụng không gây hại cho người bệnh và nhân viên y tế. Mở rộng kiến thức về vật tư tiêu hao, linh kiện và các công nghệ mới trong ngành thiết bị y tế. \\[0.3 cm]
Về kỹ năng chuyên môn, thực hành các công việc như kiểm tra, bảo trì, và hỗ trợ lắp đặt thiết bị. Được rèn luyện kỹ năng làm việc nhóm, giao tiếp hiệu quả với đồng nghiệp, và khả năng giải quyết các vấn đề kỹ thuật phát sinh trong quá trình làm việc. Kế đến là phân tích, tổng hợp thông tin từ các tài liệu kỹ thuật để hỗ trợ quá trình làm việc. \\[0.3 cm]
Về thái độ, em luôn cố gắng chủ động học hỏi, tìm tòi thêm kiến thức từ anh chị trong công ty cũng như từ tài liệu chuyên ngành. Thực hiện các nhiệm vụ được giao với tinh thần trách nhiệm, thái độ nghiêm túc và sự cẩn trọng. Đồng thời, em chú trọng việc tuân thủ các quy định, nội quy của công ty, xây dựng tác phong làm việc chuyên nghiệp và tích cực.\\[0.3 cm]
\section{Ý nghĩa của đợt thực tập}
Đợt thực tập này không chỉ mang ý nghĩa đối với cá nhân em mà còn đem lại giá trị thiết thực cho cả nhà trường và doanh nghiệp. Với bản thân em, đây là bước chuyển quan trọng từ kiến thức lý thuyết trên giảng đường sang thực tiễn công việc. Quá trình thực tập giúp em xác định rõ định hướng nghề nghiệp, nhận ra điểm mạnh và những mặt còn hạn chế để tiếp tục rèn luyện. Đồng thời, em có cơ hội giao lưu, học hỏi từ các chuyên gia trong ngành, tạo nền tảng thuận lợi cho con đường nghề nghiệp sau này.\\[0.3 cm]
Đối với nhà trường, báo cáo thực tập là cơ sở đánh giá chất lượng đào tạo, từ đó điều chỉnh chương trình giảng dạy cho phù hợp hơn với yêu cầu thực tế. Việc hợp tác với doanh nghiệp cũng góp phần gắn kết mối quan hệ giữa hai bên, đồng thời mở rộng cơ hội thực tập và việc làm cho các thế hệ sinh viên tiếp theo.\\[0.3 cm]
Với doanh nghiệp, sinh viên thực tập là nguồn nhân lực trẻ, có thể hỗ trợ một số công việc cơ bản, góp phần tạo môi trường làm việc năng động, sáng tạo. Đây cũng là dịp để doanh nghiệp tìm kiếm và bồi dưỡng những nhân sự tiềm năng trong tương lai.
Nội dung của đợt thực tập \\[0.3 cm]
Trong suốt thời gian thực tập tại công ty, em tập trung tìm hiểu một cách hệ thống về nguyên lý hoạt động, cấu tạo, đặc tính kỹ thuật và quy trình vận hành của các thiết bị chuyên dụng trong lĩnh vực y tế. Song song với đó, em cũng được tiếp cận và nghiên cứu các loại vật tư thường sử dụng trong thực tế, từ những thiết bị quan trọng như máy cắt tiêu bản cho đến nhiều loại vật tư tiêu hao khác. Việc quan sát, tìm hiểu và làm quen trực tiếp với các thiết bị này không chỉ giúp em củng cố kiến thức lý thuyết đã học trên giảng đường, mà còn mang đến cái nhìn thực tế về cách chúng được ứng dụng trong công tác chuyên môn, qua đó nâng cao sự hiểu biết và kỹ năng thực hành của bản thân. \\[0.3 cm]