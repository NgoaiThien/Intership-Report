\section{Bể dàn tiêu bản}
\subsection{Giới thiệu chung}
Bể dàn tiêu bản TFB (35,45,55) là các bể dàn tiêu bản hình tròn được sản xuất bởi hãng MEDITE, một thương hiệu uy tín lâu đời từ Đức chuyên về thiết bị giải phẫu mô bệnh học.\\
Đây là một thiết bị nhỏ gọn và giá cả phải chăng được làm từ nhôm phủ bột với dung tích 1,5 lít. Các lát cắt parafin có thể được kéo giãn và hút lên phiến kính.Lớp phủ nhựa màu đen sâu cung cấp độ tương phản tối ưu với lát cắt parafin và đảm bảo tuổi thọ lâu dài.Nhờ trọng lượng thấp và kích thước nhỏ gọn, thiết bị có thể được sử dụng linh hoạt tại nhiều nơi làm việc khác nhau trong các phòng thí nghiệm bệnh lý, mô học và các phòng thí nghiệm khác.\cite{Catalog_TFB_45_55} \\
\subsection{Mục đích sử dụng} TFB 55 được sản xuất với mục đích nhúng các mẫu mô trong các phòng thí nghiệm mô học và bệnh học.\cite{Catalog_TFB_45_55}\\
Bể nổi mô được yêu cầu sau bước cắt các lát parafin và trước khi đặt chúng lên phiến kính. Nó cho phép các mô thư giãn và làm mịn trước khi được gắn lên phiến kính cũng như giúp parafin bám dính vào phiến kính. Điều này đảm bảo loại bỏ các nếp nhăn và nếp gấp trước khi các lát cắt được đặt lên phiến kính.

\subsection{Cấu tạo}
    \begin{figure}[H]
    \centering
    \includegraphics[width=0.8\linewidth]{TFB 45.png}
    \caption{Bể dàn tiêu bản TFB 45 \cite{Catalog_TFB_45_55}}
    \label{fig:TFB45}
    \end{figure}
    \begin{figure}[H]
    	\centering
    	\includegraphics[width=0.8\linewidth]{TFB 55.png}
    	\caption{Bể dàn tiêu bản TFB 55 \cite{Catalog_TFB_45_55}}
    	\label{fig:TFB55}
    \end{figure}
\subsection{Thông số kỹ thuật} \cite{Catalog_TFB_45_55}
\begin{table}[H]
	\centering
	\renewcommand{\arraystretch}{1.4} % Tăng chiều cao dòng cho dễ đọc
	\begin{tabular}{|p{5.5cm}|p{10cm}|}
		\hline
		% Dòng tiêu đề màu xanh
		 
		\multicolumn{2}{|l|}{\bfseries\color{black} Thông số kĩ thuật \quad TFB 45} \\ 
		\hline
		Dung tích bồn tắm: & tối đa 1.5 lít \\ \hline
		Nhiệt độ: & +30 $^\circ$C đến +90 $^\circ$C \\ \hline
		Chiều rộng vành: & 45 mm \\ \hline
		Kích thước phần bên trong (Ø/H): & 210 x 55 mm \\ \hline
		Kích thước tổng thể (Ø/H): & 300 x 90 mm \\ \hline
		Cân nặng: & ca. 2 kg \\ \hline
		Nguồn điện: & 230 V / 50 – 60 Hz / 250 VA $\rightarrow$ Cat. No. 01-4500-00 \newline 115 V / 50 – 60 Hz / 250 VA $\rightarrow$ Cat. No. 01-4510-00 \\ \hline
	\end{tabular}
\end{table}

\vspace{0.5cm} % Khoảng cách giữa 2 bảng

% --- BẢNG 2: TFB 55 ---
\begin{table}[H]
	\centering
	\renewcommand{\arraystretch}{1.4}
	\begin{tabular}{|p{5.5cm}|p{10cm}|}
		\hline
		% Dòng tiêu đề màu xanh
	 
		\multicolumn{2}{|l|}{\bfseries\color{black} Thông số kĩ thuật \quad TFB 55} \\ 
		\hline
		Dung tích bồn tắm: & tối đa 1.5 lít \\ \hline
	
		 Nhiệt độ: & +30 $^\circ$C đến +90 $^\circ$C \\ \hline
		Chiều rộng vành: & 55 mm \\ \hline
		Kích thước phần bên trong (Ø/H): & 210 x 55 mm \\ \hline
		Kích thước tổng thể (Ø/H): & 340 x 90 mm \\ \hline
		Cân nặng: & ca. 2 kg \\ \hline
		Nguồn điện: & 230 V / 50 – 60 Hz / 250 VA $\rightarrow$ Cat. No. 01-5500-00 \newline 115 V / 50 – 60 Hz / 250 VA $\rightarrow$ Cat. No. 01-5510-00 \\ \hline
	\end{tabular}
\end{table}
\subsection{Danh sách phụ kiện đi kèm}
\begin{longtable}{|p{9cm}|c|c|}
	\hline
	\textbf{Hạng mục / Tên thiết bị} & \textbf{Đơn vị đóng gói} & \textbf{Mã số (Cat. No.)} \\ \hline
	% Hàng 1
	Bộ phụ kiện cho \textbf{TFB 45} bao gồm nắp, nhiệt kế và giá đỡ phù hợp & 1 bộ & 11-4500-00 \\ \hline
	% Hàng 2
	Bộ phụ kiện cho \textbf{TFB 55} bao gồm nắp, nhiệt kế và giá đỡ phù hợp & 1 bộ & 11-5500-00 \\ \hline
	% Hàng 3
	Nhiệt kế cho \textbf{TFB 45} và \textbf{TFB 55}, $0 - 100^\circ C$ & 1 cái & 51-4551-00 \\ \hline
	% Hàng 4
	Nắp làm bằng nhôm cho \textbf{TFB 45} và \textbf{TFB 55} & 1 cái & 11-4540-00 \\ \hline
	% Hàng 5
	Giá đỡ nhiệt kế cho \textbf{TFB 45} & 1 cái & 11-4541-00 \\ \hline
	% Hàng 6
	Giá đỡ nhiệt kế cho \textbf{TFB 55} & 1 cái & 11-5541-00 \\ \hline
	
\end{longtable}
\subsubsection{Hình ảnh một số phụ kiện đi kèm}
\begin{figure}[H]
	\begin{subfigure}{.4\textwidth}
		\centering
		\includegraphics{lid.png}
		\caption{Nắp làm bằng nhôm}
	\end{subfigure}
	\hfill
	\begin{subfigure}{.4\textwidth}
		\centering
		\includegraphics{thermometer.png}
		\caption{Nhiệt kế}.
	\end{subfigure}
	\hfill
	\begin{subfigure}{.4\textwidth}
		\centering
		\includegraphics{thermometer2.png}
		\caption{Nắp nhiệt kế}
	\end{subfigure}
\end{figure}
\subsection{Nguyên lý hoạt động}
\subsubsection{Nguyên lí gia nhiệt và truyền nhiệt}
\begin{itemize}
	\item \textbf{Chuyển hóa năng lượng}: Khi thiết bị được cấp nguồn điện (230V hoặc 115V), dòng điện chạy qua hệ thống điện trở nhiệt (heating elements) được tích hợp bên dưới và xung quanh lòng bể kim loại. Tại đây, điện năng được chuyển hóa thành nhiệt năng.
	\item \textbf{Dẫn nhiệt} : Nhiệt lượng từ điện trở được truyền trực tiếp vào vỏ kim loại của lòng bể (basin) và phần vành đai mở rộng . Do được làm bằng vật liệu dẫn nhiệt tốt (thường là nhôm), nhiệt độ được phân bố nhanh và đồng đều trên toàn bộ bề mặt kim loại.
	\item \textbf{Đối lưu nhiệt} : Nhiệt từ lòng bể kim loại truyền sang khối nước chứa bên trong. Nước nóng có xu hướng di chuyển lên trên và nước lạnh chìm xuống dưới, tạo thành dòng đối lưu tự nhiên giúp nhiệt độ nước đồng nhất tại mọi điểm trong bể, duy trì ổn định ở mức cài đặt (thường từ 40°C - 50°C tùy loại sáp).
\end{itemize}
\subsubsection{Cơ chế làm phẳng mẫu}
\begin{itemize}
	\item \textbf{Tác động nhiệt}: Khi lát cắt mô nến từ máy cắt được thả vào bể nước ấm, nhiệt độ của nước sẽ làm mềm sáp paraffin bao quanh mô.
	\item \textbf{Sức căng bề mặt} : Sự kết hợp giữa việc sáp mềm ra và sức căng bề mặt của nước sẽ tạo ra lực kéo, giúp lát cắt tự động giãn phẳng, loại bỏ các nếp nhăn  hoặc các vết gấp hình thành do lực nén của dao cắt.
\end{itemize}
\subsection{Lắp đặt}
\cite{instru_TFB55}
\begin{enumerate}
	\item Đặt giá đỡ nhiệt kế với nhiệt kế thủy tinh lên vành của bể, ở phía đối diện với bảng điều khiển, đảm bảo rằng nhiệt kế không chạm vào đáy bể.\\
	(Việc cố định và điều chỉnh chiều cao của nhiệt kế trên giá đỡ được thực hiện bằng cách cuộn vòng cao su đi kèm lên nhiệt kế từ phía trên)
	\item Đổ nước cất hoặc nước đã khử muối hoàn toàn vào bể cho đến khi mực nước cách vành bể 1-2 cm. Nếu sử dụng nước máy, bề mặt bên trong màu đen sẽ trở nên trắng do lắng đọng cặn vôi. Khi đó, các lát cắt sẽ không còn nhìn rõ nữa.
	\item Trước khi kết nối thiết bị với nguồn điện bằng dây nguồn đi kèm vào nguồn điện đã nối đất, cần kiểm tra xem nguồn điện có đúng như thông số ghi trên nhãn của thiết bị hay không.
	\item Kết nối thiết bị với nguồn điện chính.
	\item Bật công tắc chính màu xanh lá cây.
	\item Chọn nhiệt độ bằng cách xoay núm điều chỉnh nhiệt độ (luôn xoay theo chiều kim đồng hồ). Đèn báo sẽ sáng lên và tắt khi đạt đến nhiệt độ cài đặt.
\end{enumerate}


Trước khi đạt đến nhiệt độ yêu cầu, hệ thống sưởi sẽ tự động tắt bởi bộ điều chỉnh nhiệt độ trong thời gian ngắn vài lần. Điều này giúp tránh hiện tượng quá nhiệt.\\

\subsection{Quy tình vận hành}
\cite{instru_TFB55}
\subsubsection{Chuẩn bị và bật nguồn}
\begin{enumerate}
	\item Đặt giá đỡ nhiệt kế có nhiệt kế thủy tinh lên vành bồn tắm đối diện với 
	bảng điều khiển. (Việc cố định và điều chỉnh độ cao của nhiệt kế trên giá đỡ 
	nhiệt kế được thực hiện bằng vòng cao su kèm theo được cuộn trên nhiệt 
	kế từ trên xuống.
	\item  Đổ đầy nước cất hoặc nước đã khử muối hoàn toàn đến 1-2cm dưới vành 
	bồn tắm. Nếu sử dụng nước máy, bề mặt bên trong màu đen sẽ trở thành 
	màu trắng do cặn vôi. các phần sau đó không được nhìn thấy rõ ràng nữa. 
	\item Bật thiết bị bằng cách ấn công tắc nguồn về vị trí “I” ở mặt trước của thiết bị.
	\item  Chọn nhiệt độ bằng cách quay nút vặn chỉnh nhiệt độ theo chiều kim đồng 
	hồ để đến vị trí nhiệt độ mong muốn đèn sẽ sáng lên.
\end{enumerate}

\subsubsection{Tắt nguồn}
Nhấn công tắc nguồn vị trí "0" để tắt TFB 55.

\subsection{Bảo dưỡng}
\cite{instru_TFB55}
Chỉ sử dụng các chất tẩy rửa nhẹ để làm sạch thiết bị, vì bột tẩy rửa mạnh có thể làm hỏng lớp phủ epoxy màu đen.

Trong trường hợp thiết bị không hoạt động (ví dụ do dao động điện áp), hãy kiểm tra xem các cầu chì ở phía sau thiết bị có bị hỏng không.\\

