\chapter{Kết Luận}
Trong bối cảnh khoa học công nghệ phát triển không ngừng, vai trò của các máy móc, thiết bị y tế trong các viện nghiên cứu, bệnh viện và cơ sở y tế ngày càng trở nên thiết yếu. Chúng không chỉ là công cụ hỗ trợ mà còn là nền tảng cốt lõi cho mọi hoạt động từ chẩn đoán, điều trị đến nghiên cứu chuyên sâu. Các thiết bị hiện đại góp phần loại bỏ những sai sót chủ quan của con người, mang lại dữ liệu chính xác và khách quan – yếu tố đặc biệt quan trọng trong chẩn đoán bệnh lý, nơi mà một sai lệch nhỏ có thể dẫn đến hậu quả nghiêm trọng. Hơn nữa, việc tự động hóa các quy trình bằng máy móc giúp rút ngắn thời gian thao tác, nâng cao năng suất và cho phép xử lý khối lượng mẫu lớn, phục vụ hiệu quả cho nhiều bệnh nhân hơn.\\
 Đặc biệt, đối với quy trình Giải phẫu bệnh, sự kết hợp giữa TFB 45/55 và OTS 40 mang lại giải pháp chuẩn bị mẫu toàn diện. Với khả năng kiểm soát nhiệt độ chính xác và thiết kế tối ưu cho thao tác của kỹ thuật viên, bộ đôi này đảm bảo các lát cắt mô đạt chất lượng bề mặt phẳng mịn và độ bám dính tối đa, tạo tiền đề quan trọng cho kết quả chẩn đoán mô học chính xác.
 \\Tuy nhiên, điểm nhấn công nghệ mang tính đột phá được trình bày trong báo cáo này nằm ở hệ thống đo chức năng hô hấp SpiroScout. Khác biệt hoàn toàn với các thế hệ máy đo truyền thống dựa vào tuabin cơ học, SpiroScout ứng dụng công nghệ cảm biến siêu âm tiên tiến giúp loại bỏ hoàn toàn sai số do quán tính vật lý hay độ trễ, đảm bảo độ chính xác tuyệt đối ngay cả khi đo ở những dải lưu lượng khí thấp nhất. Giá trị cốt lõi của thiết bị không chỉ dừng lại ở năng lực đo lường vượt trội mà còn giải quyết bài toán vận hành tối ưu thông qua cơ chế hoạt động không cần hiệu chuẩn, giúp cơ sở y tế tiết kiệm đáng kể thời gian và chi phí bảo trì. Đặc biệt, trong bối cảnh các bệnh lý hô hấp lây nhiễm ngày càng phức tạp, thiết kế chú trọng kiểm soát nhiễm khuẩn của SpiroScout với các bộ lọc PFT và vật tư tiêu hao dùng một lần trở thành lá chắn an toàn vững chắc, ngăn ngừa tuyệt đối nguy cơ lây nhiễm chéo giữa các bệnh nhân. Tựu trung lại, SpiroScout không chỉ đơn thuần là một công cụ đo lường mà thực sự là "bộ não" tinh vi, đại diện cho xu hướng hiện đại hóa trong chẩn đoán chức năng phổi, góp phần nâng cao uy tín và chất lượng điều trị của đơn vị.